%!TEX TS-program = xelatex
%!TEX encoding = UTF-8 Unicode
% Awesome CV LaTeX Template for CV/Resume
%
% This template has been downloaded from:
% https://github.com/posquit0/Awesome-CV
%
% Original author:
% Claud D. Park <posquit0.bj@gmail.com>
% http://www.posquit0.com
%
% Modifications by:
% Junhao Dong <junhao.dong96@gmail.com>
%
% Template license:
% CC BY-SA 4.0 (https://creativecommons.org/licenses/by-sa/4.0/)
%


%-------------------------------------------------------------------------------
% CONFIGURATIONS
%-------------------------------------------------------------------------------
% A4 paper size by default, use 'letterpaper' for US letter
\documentclass[11pt, letterpaper]{awesome-cv}

% Configure page margins with geometry
\geometry{left=1.4cm, top=.8cm, right=1.4cm, bottom=1.8cm, footskip=.5cm}

% Specify the location of the included fonts
\fontdir[fonts/]

% Color for highlights
% Awesome Colors: awesome-emerald, awesome-skyblue, awesome-red, awesome-pink, awesome-orange
%                 awesome-nephritis, awesome-concrete, awesome-darknight
% OSU orange: osuorange
% beaver orange: beaverorange
\colorlet{awesome}{beaverorange}
% Uncomment if you would like to specify your own color
% \definecolor{awesome}{HTML}{CA63A8}

% Colors for text
% Uncomment if you would like to specify your own color
% \definecolor{darktext}{HTML}{414141}
% \definecolor{text}{HTML}{333333}
% \definecolor{graytext}{HTML}{5D5D5D}
% \definecolor{lighttext}{HTML}{999999}

% Set false if you don't want to highlight section with awesome color
\setbool{acvSectionColorHighlight}{true}

% If you would like to change the social information separator from a pipe (|) to something else
\renewcommand{\acvHeaderSocialSep}{\quad\textbar\quad}

\makeatletter
\patchcmd{\@sectioncolor}{\color}{\mdseries\color}{}{}
\makeatother

% extra packages
\usepackage{makecell}
\usepackage{lastpage}

%-------------------------------------------------------------------------------
%	PERSONAL INFORMATION
%	Comment any of the lines below if they are not required
%-------------------------------------------------------------------------------
% Available options: circle|rectangle,edge/noedge,left/right
% \photo[rectangle,edge,right]{profile}
\name{Austin}{Warren}
%\position{Software Engineer}
\address{149 Wheeler St., Lebanon, OR 97355}

\mobile{253-549-9420}
\email{warren.austinm@gmail.com}
%\email{warrenau@oregonstate.edu}
% \homepage{homepage}
\github{warrenau}
\linkedin{austinwarren}
% \gitlab{gitlab-id}
% \stackoverflow{SO-id}{SO-name}
% \twitter{@twit}
% \skype{skype-id}
% \reddit{reddit-id}
%\extrainfo{warrenau@oregonstate.edu}
\quote{I am an NEUP Fellow at Oregon State University looking to perform my required 10 week internship with TerraPower over Spring or Summer 2023. I have experience with software development, CFD, fluid mechanics, and heat transfer. I have also built, maintained, and ran experimental facilities. I am interested in both experimental and computational work and can learn new systems, facilities, or software quickly. My background is mostly in Thermal Hydraulics, but I have some experience with Neutronics and Fuels also. I work well with a team or on my own and can present my work clearly to both technical and layman audiences.}

%-------------------------------------------------------------------------------
\begin{document}

% Print the header with above personal informations
% Give optional argument to change alignment(C: center, L: left, R: right)
\makecvheader[C]

% Print the footer with 3 arguments(<left>, <center>, <right>)
% Leave any of these blank if they are not needed
 \makecvfooter
   {}
   {\thepage/ \pageref{LastPage}}
   {}

%-------------------------------------------------------------------------------
%	CV/RESUME CONTENT
%	Each section is imported separately, open each file in turn to modify content
%-------------------------------------------------------------------------------
\cvsection{Education}

\begin{cventries}
  \cventry
    {\makecell[l]{Master of Science (M.S.) in Nuclear Engineering}} % Degree % Cambria broke the dash formatting; have to use \textendash or \textemdash for longer dashes instead of -- or ---
    {Oregon State University} % Institution
    {Corvallis, OR} % Location
    {\makecell[r]{Expected March 2023}} % Date(s)
    {
      \begin{cvitems} % Description(s) bullet points
         \item{Thesis: \textit{Validation Experiment Design for Coupled CFD-Reactor Physics Model CONSTELATION for Helium-3 Injection System HENRI}}
      \end{cvitems}
    }

  \cventry
    {\makecell[l]{Bachelor of Science (B.S.) in Nuclear Engineering \textemdash \ Magna Cum Laude}} % Degree % Cambria broke the dash formatting; have to use \textendash or \textemdash for longer dashes instead of -- or ---
    {Oregon State University} % Institution
    {Corvallis, OR} % Location
    {\makecell[r]{June 2020}} % Date(s)
    {
      \begin{cvitems} % Description(s) bullet points
         \item{Minor in Exercise Physiology}
         \vspace{0.5mm}
         \item{Dean's List \textemdash \ \entrydatestyle{Spring 2019, Fall 2019, Winter 2020, Spring 2020}}
         \vspace{0.5mm}
         \item {Alpha Nu Sigma Member}
         %\textbf{University of Hong Kong}, Hong Kong, China --- \entrydatestyle{Semester Abroad Fall 2018}}
         \vspace{0.5mm}
         \item{American Nuclear Society Student Chapter Member}
         \vspace{0.5mm}
         \item {Rowing Team Walk-On \textemdash \ \entrydatestyle{2015}}
      \end{cvitems}
    }
    
    
    %%% This makes the two degrees be in the same entry
    % {\makecell[l]{Master of Science (M.S.) in Nuclear Engineering \textemdash \ GPA: 3.71\\ Bachelor of Science (B.S.) in Nuclear Engineering \textemdash \ GPA: 3.71, \ Magna Cum Laude}} % Degree % Cambria broke the dash formatting; have to use \textendash or \textemdash for longer dashes instead of -- or ---
    % {Oregon State University} % Institution
    % {Corvallis, OR} % Location
    % {\makecell[r]{Expected June 2022\\ June 2020}} % Date(s)
    % {
    %   \begin{cvitems} % Description(s) bullet points
    %      \item{Minor in Exercise Physiology}
    %      \vspace{0.5mm}
    %      \item{Dean's List \textemdash \ \entrydatestyle{Spring 2019, Fall 2019, Winter 2020, Spring 2020}}
    %      \vspace{0.5mm}
    %      \item {Alpha Nu Sigma Member}
    %      %\textbf{University of Hong Kong}, Hong Kong, China --- \entrydatestyle{Semester Abroad Fall 2018}}
    %      \vspace{0.5mm}
    %      \item{American Nuclear Society Student Chapter Member}
    %      \vspace{0.5mm}
    %      \item {Rowing Team Walk-On \textemdash \ \entrydatestyle{2015}}
    %   \end{cvitems}
    % }
\end{cventries}

\cvsection{Awards, Fellowships, and Grants}

% \begin{cventries}

%     \cventry
%     {Graduate Fellowship} % 
%     {Nuclear Energy University Program (NEUP)} % institution / organization
%     {Oregon State University} % location -- prolly dont need
%     {2021-2024} % date -- maybe need, maybe put with individual awards
%     {
%     \begin{cvitems}
%         \item{Awarded based on NEUP application process, which evaluates research and career goals, along with research experience.}
%     \end{cvitems}
%     }
    
    
%     \cventry
%     {Henry W. \& Janice J. Schuette Graduate Fellowship} % 
%     {Oregon State University} % institution / organization
%     {Oregon State University} % location -- prolly dont need
%     {2020-2021} % date -- maybe need, maybe put with individual awards
%     {
%     \begin{cvitems}
%         \item{Awarded through university scholarship and fellowship application process.}
%     \end{cvitems}
%     }
    
    




% \end{cventries}


\begin{cvhonors}

    \cvhonor
    {Graduate Fellowship}% position
    {Nuclear Energy University Program (NEUP) Graduate Fellowship}% title
    {Oregon State University}% location
    {2021-2024}% date
    
    \cvhonor
    {Graduate Fellowship}% position
    {Henry W. \& Janice J. Schuette Graduate Fellowship}% title
    {Oregon State University}% location
    {2021}% date

\end{cvhonors}
\cvsection{Skills}

\begin{cvskills}
  \cvskill
    {Languages} % Type
    {Python, MATLAB, LabVIEW, Serpent 2, MCNP} % Skillset

  \cvskill
    {Programs} % Type
    {STAR-CCM+, SolidWorks, AutoCAD, Inventor, NX} % Skillset
    
  \cvskill
    {Other} % Type
    {\LaTeX, Git, GitHub, Home Server / Home Networking} % Skillset
\end{cvskills}

\cvsection{Experience}

\begin{cventries}
  \cventry
    {Graduate Fellow} % Job title
    {Oregon State University} % Organization
    {Corvallis, OR} % Location
    {September 2021 - Present} % Date(s)
    {
      \begin{cvitems} % Description(s) of tasks/responsibilities
        \item {Took over development of CONSTELATION and modified for use at the OSU TRIGA reactor.}
        \item{Developed astronomer package to process CONSTELATION outputs.}
        \item{Designed experiments to validate CONSTELATION using OSU TRIGA reactor.}
        \item{Developed STAR-CCM+ model for designed experiments.}
        \item{Delegated work to undergraduate research assistants.}
        \item{Presented work at conferences.}
      \end{cvitems}
    }


  \cventry
    {HENRI Design Intern} % Job title
    {Idaho National Laboratory} % Organization
    {Idaho Falls, ID} % Location
    {Summer 2019, 2020, and 2021} % Date(s)
    {
      \begin{cvitems} % Description(s) of tasks/responsibilities
        \item {Defined physical dimensions, limits, and parameters for the Helium-3 Enhanced Negative Reactivity Insertion (HENRI) prototype for the Transient Reactor Test (TREAT) facility.}
        \item{Developed concept for a helium recovery system to be used with HENRI.}
        \item{Modeled HENRI prototypes in TREAT using MCNP.}
        \item{Processed CFD data into a format to be used by MCNP for HENRI reactivity calculations in TREAT.}
        \item{Proposed design for HENRI experiments in the OSU TRIGA reactor.}
      \end{cvitems}
    }

  \cventry
    {Undergraduate Research Assistant} % Job title
    {Oregon State University School of Nuclear Science and Engineering} % Organization
    {Corvallis, OR} % Location
    {June 2018 - June 2020} % Date(s)
    {
      \begin{cvitems} % Description(s) of tasks/responsibilities
        \item {Worked with Quality Assurance documents and procedures in a NQA-1 compliant organization.}
        \item{Assembled and maintained experimental facilities.}
        \item{Designed and implemented LabVIEW code for HENRI to display and record test data.}
        \item{Executed tests on HENRI and the Naval Reactor Test Loop (NRTL).}
        %\item{Co-authored paper for NURETH-18 conference.}
        \item{Presented project progress to sponsors.}
      \end{cvitems}
    }

  \cventry
    {Tutor} % Job title
    {Oregon State University Academics for Student Athletes} % Organization
    {Corvallis, OR} % Location
    {January 2017 - January 2020} % Date(s)
    {
      \begin{cvitems} % Description(s) of tasks/responsibilities
        \item {Tutored student athletes in math, physics, chemistry, engineering, English, writing, and other subjects.}
        \item{Taught to different learning styles and varying levels of understanding.}
      \end{cvitems}
    }

  \cventry
    {Coach} % Job title
    {Gig Harbor Canoe and Kayak Racing Team} % Organization
    {Gig Harbor, WA} % Location
    {March 2012 - September 2017} % Date(s)
    {
      \begin{cvitems} % Description(s) of tasks/responsibilities
        \item {Coached team at five National Championship wins, as well as many regional regattas in the US and Canada.}
		\item {Designed workouts for athletes, catered to experience, age, and goals.}
		\item{Supervised up to 30 athletes at any given time.}
      \end{cvitems}
    }

  
\end{cventries}

\cvsection{Projects}

\begin{cventries}
  \cventry
    {Current Developer}      % Position
    {CONSTELATION}           % Project
    {Oregon State University}% Location
    {June 2022-Present}   % Date
    {
    \begin{cvitems}
        \item{Coupled model for HENRI using STAR-CCM+ and Serpent 2 with Python wrapper.}
        \item{Modified for use in OSU TRIGA reactor.}
        \item{In development: \href{https://github.com/warrenau/CONSTELATION/tree/ostr}{https://github.com/warrenau/CONSTELATION}.}
    \end{cvitems}
    }


  \cventry
    {Creator}                % Position
    {\textit{astronomer}}             % Project
    {Oregon State University}% Location
    {January 2022-Present}   % Date
    {
    \begin{cvitems}
        \item{Software package written in Python to process outputs of coupled model CONSTELATION for HENRI.}
        \item{In development: \href{https://github.com/warrenau/astronomer}{https://github.com/warrenau/astronomer}.}
    \end{cvitems}
    }

  
  \cventry
    {Team Leader} % Empty position
    {XeNRI} % Project
    {Oregon State University} % Empty location
    {2020} % Empty date
    {
      \begin{cvitems} % Description(s) bullet points
        \item{A senior design project that supports the HENRI system.}
		\item {Uses xenon-135, instead of helium-3, to characterize the HENRI system using the Oregon State University TRIGA reactor.}
		\item {Won project of the year within the School of Nuclear Science and Engineering.}
		\item{Presented as a member of a Student Design Project Panel with OSU's College of Engineering.}
        \item{More information can be found at the website: \href{https://sites.google.com/oregonstate.edu/xenridesignproject/home}{https://sites.google.com/oregonstate.edu/xenridesignproject/home}.}
      \end{cvitems}
    }


    \cventry
    {Research Assistant} % Empty position
    {HENRI} % Project
    {Oregon State University} % Empty location
    {June 2018 - Present} % Empty date
    {
      \begin{cvitems} % Description(s) bullet points
        \item{Helium-3 Enhanced Negative Reactivity Insertion (HENRI) system for Transient Reactor Test (TREAT) Facility, with out-of-pile prototype built at OSU.}
        \item{Assembled facility and ran experiments.}
        \item{Designed and assembled control panel for instrumentation and control.}
        \item{Developed control software using LabVIEW.}
      \end{cvitems}
    }


\end{cventries}
\cvsection{Publications}

\cvsubsection{Conferences}

\begin{cventries}

%%%%%%%%%%%%%%%%%%%%%%%%%%%%%%%%%%%%%%%%%%%%%%%%%%%%%%%%%%%
    \cventry
    {Abstract Accepted} % position ?
    {20th International Topical Conference on Nuclear Reactor Thermal Hydraulics (NURETH-20)} % conference / journal
    {Washington D.C., USA} % location
    {August 20-25, 2023} % date
    {
    \begin{cvitems}
        \item{Primary and presenting author for \textit{Scaled Experiment Design for Transient Coupled CFD-Reactor Physics Model for HENRI}}
    \end{cvitems}
    }


%%%%%%%%%%%%%%%%%%%%%%%%%%%%%%%%%%%%%%%%%%%%%%%%%%%%%%%%%%%
    \cventry
    {Attended} % position ?
    {2022 American Nuclear Society Winter Conference and Technology Expo} % conference / journal
    {Phoenix, Arizona} % location
    {November 13-17, 2022} % date
    {
    \begin{cvitems}
        \item{Primary and presenting author for \textit{Validation Experiment Design for HENRI Coupled Code CONSTELATION}}
    \end{cvitems}
    }

%%%%%%%%%%%%%%%%%%%%%%%%%%%%%%%%%%%%%%%%%%%%%%%%%%%%%%%%%%%
    \cventry
    {Attended virtually} % position ?
    {19th International Topical Conference on Nuclear Reactor Thermal Hydraulics (NURETH-19)} % conference / journal
    {Brussels, Belgium} % location
    {March 6-11, 2022} % date
    {
    \begin{cvitems}
        \item{Primary and presenting author for \textit{Experiments on the Helium-3 Negative Reactivity Insertion (HENRI) Prototype and Fast Opening Valve}}
    \end{cvitems}
    }

%%%%%%%%%%%%%%%%%%%%%%%%%%%%%%%%%%%%%%%%%%%%%%%%%%%%%%%%%%%
    \cventry
    {Attended virtually} % position ?
    {American Nuclear Society Student Conference 2021} % conference / journal
    {Rayleigh, North Carolina} % location
    {March 2021} % date
    {
    \begin{cvitems}
        \item{Primary and presenting author for \textit{Experiment Design for Code Coupling Validation for the Helium-3 Enhanced Negative Reactivity Insertion (HENRI) System} (Summary)}
    \end{cvitems}
    }

%%%%%%%%%%%%%%%%%%%%%%%%%%%%%%%%%%%%%%%%%%%%%%%%%%%%%%%%%%%
    \cventry
    {Attended} % position ?
    {18th International Topical Conference on Nuclear Reactor Thermal Hydraulics (NURETH-18)} % conference / journal
    {Portland, Oregon} % location
    {August 2019} % date
    {
    \begin{cvitems}
        \item{Contributing author for \textit{Experiments on Helium-3 Negative Reactivity Insertion -HENRI- Prototype}}
    \end{cvitems}
    }






\end{cventries}

%-------------------------------------------------------------------------------
\end{document}
